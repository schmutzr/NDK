\documentclass[ignorenonframetext]{beamer}
\usepackage{beamerthemesplit}
\usepackage{amssymb}

%\documentclass[a4paper]{article}
\usepackage{beamerarticle}
\usepackage{verbatim}

\usepackage{../fhnw-beamer}

%\mode<article>{\usepackage{fullpage}}
%\mode<presentation>{\usetheme{Berlin}}

\date{\today}
\author{rolf.schmutz@fhnw.ch}
\institute{FHNW}
\title {Netzwerke und Datenkommunikation\\B-LS-MI 004\\Physical Layer}


\begin{document} % ===============================================================

\section{NDK B-LS-MI 004: L2}



\begin{frame}
\titlepage
\end{frame}

\begin{frame}
\frametitle{Ziele}
\begin{itemize}
	\item{Repr\"asentation des Quellsignals auf elektromagnetischer Ebene}
	\item{Codierung des Quellsignals (Abgek\"urzt)}
	\item{Verfahren zur Leitungscodierung der Daten aus einem Quellenstrom}
	\item{Techniken in Bezug auf Basisband- und Breitband-Kommunikation (Modulation)}
	\item{Fehlererkennung und -Korrektur}
\end{itemize}
\end{frame}





\begin{frame}
\frametitle{Einfachste Bit-Serielle Daten\"ubertragung}
\includegraphics{simplest-serial}

\begin{itemize}
  \item ``Hello World!'' soll \"ubertragen werden
\end{itemize}
\end{frame}
%% Linu/URL/referenz: \myurl{http://www.rfc-editor.org/rfc/pdfrfc/rfc1918.txt.pdf}


\begin{frame}
\frametitle{Probleme}
\begin{itemize}
  \item wie wird ``Hello World!'' als Abfolge von Licht/kein-Licht (0, 1) dargestellt? (Quellcodierung)
  \item wann beginnt die Nachricht, einzelne Buchstaben, einzelne Bits, wann enden sie?
  \item wie k\"onnen einzelne gleiche ``bits'' getrennt werden? z.B. ``o''=01101111
\end{itemize}
\end{frame}


\begin{frame}
\frametitle{Quellencodierung (source-coding) 1/2}
\begin{block}{}
  Das ist die Repr\"asentierung von Informationen in bin\"arer (numerischer) Form, 
  also nicht Programm-Quellcode/sourcecode
\end{block}

\begin{itemize}
  \item es wird eine \"Ubereinkunft/Tabelle ben\"otigt, die die Information in numerischer Form (Bitmuster) festlegt (code-point)
  \item es gibt eine Vielzahl von Codierungen f\"ur verschiedene Datenformate
\end{itemize}
  \begin{block}{}{Die Codierung muss auf beiden Seiten bekannt sein und ist nicht gleich ``Verschl\"usselung''}\end{block}
\end{frame}

\begin{frame}
\frametitle{Quellencodierung (source-coding) 2/2}

F\"ur unsere Zwecke benutzen wir die alterw\"urdige ASCII-Codierungstabelle (ohne Kontrollzeichen):

\begin{center}
\begin{tabular}{l|l}
\begin{minipage}{5cm}
   \begin{tiny}\verbatiminput{asciitable.txt}\end{tiny}
\end{minipage} & z.B. ``H'': 48_{16} = 0100'1000_{2} \\
\end{tabular}
\end{center}

\end{frame}


\begin{frame}
\frametitle{Bit-Synchronisation: Strobe/Clock/Sampling (1/2)}
\includegraphics{simplest-serial-clock}
\end{frame}


\begin{frame}
\frametitle{Bit-Synchronisation: Strobe/Clock/Sampling (2/2)}
\begin{itemize}
  \item mit der ``Clock'' Leitung wird dem Empf\"anger der korrekte Messzeitpunkt signalisiert
  \item Folgen von ``gleichen'' Bits (alles 0 oder alles 1) k\"onnen problemlos getrennt werden
\end{itemize}
\begin{block}{Synchrone Bitserielle \"Ubertragung}
Es werden mindestens drei Leitungen ben\"otigt, daf\"ur sind keine weiteren Massnahmen n\"otig.

Synchrone Daten\"ubertragung wird vorallem im Nahbereich (im Computer, Embedded Systems I^{2}C/SPI, HDMI, etc) eingesetzt
\end{block}
\begin{itemize}
  \item es kann auch zwischen ``keine Daten'' (Clock=0) und ``0'' Bits unterschieden werden
\end{itemize}
\end{frame}


\begin{frame}
\frametitle{Asynchrone Serielle \"Ubertragung (1/3)}
Eine weitere M\"oglichkeit eine Synchronisierung\footnote{wenn auch im Titel ``Asynchron''} ist das ``Framing'' der \"Ubertragung

\begin{itemize}
\item eine Startsequenz (Startbit oder Preamble) und eine optionale Endsequenz werden in den Datenstrom eingef\"ugt\footnote{dies sind bereits keine ``Nutzdaten'' mehr sondern Teil des Protokolls}
\item der Empf\"anger hat damit die M\"oglichkeit, sich \emph{f\"ur die Dauer der Nachricht/Zeichens} mit dem Sender zu synchronisieren
\item mit dem Framing kann beim Empf\"anger auch zwischen Daten/keine-Daten unterschieden werden (ausserhalb des Frames werden Daten ignoriert)
\end{itemize}
\begin{block}{Asynchrone Bitserielle \"Ubertragung}
Es werden nur zwei Leitungen/ein Kanal ben\"otigt. Daf\"ur ist die Methode ein wenig aufwendiger zu implementieren.
\end{block}
\end{frame}



\begin{frame}
\frametitle{Asynchrone Serielle \"Ubertragung (2/3)}
Bei einfachen seriellen Schnittstellen\footnote{RS232 und \"aquivalent} wird ein Startbit (optional Stopbit) eingef\"ugt, {\bfseries jedes Byte/Zeichen wird einzeln synchronisiert}

\includegraphics[height=3cm]{asynchron-startbit}
\begin{itemize}
\item der Empf\"nger muss ungef\"ahr die Transferrate/Bitzeit schon kennen und kann das Sampling nach dem Startbit einstellen
\item die M\"oglichkeit einer Startsequenz ``10'' vereinfacht dies weiter
\item moderne Implementationen buffern die \"Ubertragung ein paar bits und k\"onnen damit ``autobaud'' -- selbst\"andige Adaption an die Datenrate implementieren
\end{itemize}
\end{frame}


\begin{frame}
\frametitle{Asynchrone Serielle \"Ubertragung (3/3)}
Bei ``Ethernet'' (der Quasi-Standard im Internet/IP-Netzwerken) wird mit einer Pr\"aambel gearbeitet
\begin{itemize}
\item 7 Bytes $AA_{16}$ + 1 Byte $AB_{16}$ {} (d.h. insgesamt 64 Bit)
\item der Empf\"anger hat eine eigene Clock-Source mit der ungef\"ahren Frequenz aber unbekannter Phase. \"Uber eine PLL wird die korrekte Phase ermittelt:
\end{itemize}
\includegraphics[height=5cm]{asynchron-ethernet}
\end{frame}


\begin{frame}
\frametitle{Interlude}
Damit w\"are das Problem ``Clock'' gel\"ost.

Wenn anstelle einer einfachen Lampe aber ein datenverarbeitendes System der
Empf\"anger ist, stellt sich eine weitere Frage:

\begin{block}{}
Wann ist ein Bit als ``1'' und wann als ``0'' zu interpretieren?
\end{block}
\end{frame}







\begin{frame}
\frametitle{Interpretation von Pegelbasierten Signalen (1/3)}
Eine \"Ubereinkunft kann mithilfe eines Schwellwerts (threshold) erreicht werden:

\includegraphics{threshold}

Dies geh\"ort nat\"urlich auch zur Protokolldefinition auf Schicht 1
\end{frame}





\begin{frame}
\frametitle{Interpretation von Pegelbasierten Signalen (2/3)}
Das Problem dabei ist eine abschw\"achung des Signals auf der Leitung (D\"ampfung), damit kann ein knappes Signal ``flackern'' und nicht eindeutig einem Wert zugewiesen werden:

\includegraphics{threshold-soso}
\end{frame}


\begin{frame}
\frametitle{Interpretation von Pegelbasierten Signalen (3/3)}
Abhilfe schafft ein sogenannter Schmitt-Trigger\footnote{in fast allen Pegelbasierten Systemen so implementiert}

\vspace{0.5cm}

\includegraphics{schmitt-trigger}
\end{frame}



\begin{frame}
\frametitle{Interlude}
Damit sind nun die Fragen:

\begin{itemize}
  \item wann soll gemessen werden?
  \item wie soll die Messung interpretiert werden?
\end{itemize}
gekl\"art.

\begin{block}{Diskriminator}
Solche Schaltungen, die ein Signal quantisieren (in zwei oder mehr Werte) werden auch {\bfseries Diskriminatoren} genannt
\end{block}
\end{frame}



\begin{frame}
\frametitle{Line-Coding Basisband/Baseband: Blockschaltbild}
\includegraphics[width=12cm]{baseband-circuit}
\begin{itemize}
  \item Source-Data sind die zu \"ubertragenden Daten. Dies kann auch eine ``analoge'' Quelle sein.
  \item Source-Encode/Decode: wie wird die Information r\"apresentiert/interpretiert. z.B. ASCII-Table
  \item Line-Code/Decode: wie wird die numerische/bin\"are Information als elektromagnetisches Signal r\"apresentiert/interpretiert
\end{itemize}
\begin{block}{}
Es muss eine \"Ubereinkunft \"uber die verwendeten Codierungen\footnote{deshalb ``nicht-geheim''} stattfinden. Line-Code geh\"ort zur Schicht 1.
\end{block}
\end{frame}


\begin{frame}
\frametitle{Baseband: Codierungen}
\begin{small}
\begin{itemize}
  \item NRZ: Basis-Schema: Gleichstromfalle
  \item NRZI: \"Anderungen nur bei einem ``1'' Bit: Transition und nicht Pegel
  \item Bipolar: kann auch ``keine Daten'' signalisieren
  \item Manchester: Flanken/Nulldurchgang codiert: Taktfrequenz ableitbar, kein Gleichstrom
\end{itemize}
\end{small}
\includegraphics[height=10.5cm]{bitcodes}

\end{frame}



\begin{frame}
\frametitle{Multiplexing/``Broadband'' (1/7)}
Bisher wurde angenommen, dass die \"Ubertragung \"uber Leitungen (galvanisch, optisch) ``exklusiv'' erfolgt.
\begin{block}{Duplex/Multiplex}
Was nun, wenn \"uber zwei Dr\"ahte mehr als ein ``Kanal'' implementiert werden soll?\footnote{z.B. Radio-/Fernsehkan\"ale Broadcast oder eine Modem-Strecke Duplex senden/empfangen auf der selben Leitung} Oder gar keine gleichstrom-f\"ahige Leitung zur Verf\"ugung steht (Radioband, Modemleitungen)
\end{block}

Dazu muss ein {\emph Multiplexing}-Verfahren eingerichtet werden. Eine solche \"Ubertragung (FDM) wird als ``Broadband'' bezeichnet.
\end{frame}




\begin{frame}
\frametitle{Multiplexing (2/7)}
\begin{columns}
\begin{column}{3cm}
\begin{center}
\includegraphics[width=0.7\textwidth]{multiplex}
\end{center}
\end{column}
\begin{column}{8cm}
\begin{itemize}
\item[TDM:] Time Division Multiplex: Die Kan\"ale werden zeitlich abwechselnd auf
	der selben Leitung \"ubertragen
\item[FDM:] Frequency Division Multiplex: Die Kan\"ale werden auf verschiedenen Frequenzen
	auf der selben Leitung \"ubertragen
\item[CDM:] Code Division Multiplex: Die Kan\"ale werden gleichzeitig mit verschiedenen
	Codes \"ubertragen
\end{itemize}
\end{column}
\end{columns}
\end{frame}







\begin{frame}
\frametitle{Multiplexing: TDM (3/7)}
Time-Division\footnote{oder ``Domain''} Multiplexing weist zeitlich getrennt (``Zeitschlitze'') einen \"Ubertragungskanal\footnote{normalerweise im Basisband, aber auch \"uber FDM-Kan\a"le \"ublich: WLAN, Funkamateure, etc} verschiedenen Quellen zu. 

Dabei kommen verschiedene Techniken zum Einsatz:
\begin{itemize}
  \item statisches Multiplexing: eine gewisse Anzahl Sender/Empf\"anger ist fest eingestellt und es wird ein ``Schalter'' f\"ur die jeweilige Paarung synchron auf Mux/Demux rotiert
  \item dynamisches Multiplexing: gleiche Voraussetzungen aber mit adaptivem Verhalten -- nicht genutzte Kan\"ale werden \"ubersprungen
  \item kooperatives Multiplexing: z.B. CSMA/CD von Ethernet: ein Sender darf nur nach einer gewissen ``Ruhephase'' des Mediums zu senden beginnen. Das ausgesendete Signal wird dabei vom Sender auf St\"orungen (ein anderer Sender) \"uberwacht und bei St\"orung abgebrochen\footnote{das f\"uhrt zu keiner deterministischen Datenbandbreite funktioniert aber im Normalfall effizienter}
\end{itemize}
\end{frame}







\begin{frame}
\frametitle{Multiplexing: FDM (4/7)}
Es werden mehrere exklusive Kan\"ale auf dem geteilten Medium eingerichtet. Dazu wird das Nutzsignal zus\"atzlich auf eine Tr\"agerwelle \emph{aufmoduliert}\footnote{z.B. Radio-/Fernsehkan\"ale, WLAN}
\vspace{0.5cm}

\includegraphics[width=12cm]{broadband-circuit}
\begin{block}{}
Die Modulation kann kontinuierlich (analog) oder in Schritten (quantisiert) erfolgen. Bei der digitalen Daten\"ubermittlung wird als ``\{Amplitude,Frequency,Phase\}-Shift-Keying'' bezeichnet.
\end{block}
\end{frame}





\begin{frame}
\frametitle{Multiplexing: FDM (5/7)}
\includegraphics[height=9cm]{modulation-schemes}
\end{frame}





\begin{frame}
\frametitle{Multiplexing: FDM (5/7)}
QAM, QPSK, APSK

Note: at any scheme (even in baseband) there might be multiple signal-levels and/or ``shifts'' to transmit more than one bit at a sampling point
\end{frame}







\begin{frame}
\frametitle{Multiplexing: FDM (6/7)}
\includegraphics[height=8cm]{fm-spectra}
\end{frame}








\begin{frame}
\frametitle{EMPTY}
\end{frame}









\begin{frame}
\frametitle{Limitation Elektromagnetischer \"Ubertragung (1/2)}
{\bfseries Jede} physikalische \"Ubertragungsstrecke unterliegt folgenden Limitationen:

\begin{itemize}
  \item verf\"ugbare physikalische Bandbreite\footnote{eigentlich aus dem Ersatzschaltbild ersichtlich} \includegraphics[height=3cm]{bandwidth}
  \item D\"ampfung des Signals (auch Frequenzabh\"angig) \includegraphics[height=3cm]{line-circuit}
\end{itemize}
\end{frame}



\begin{frame}
\frametitle{Interlude: fight the noise}
Resilienz gegen\"uber Imissionen
\end{frame}



\begin{frame}
\frametitle{Twister-Pair: Telegraphenleitung}
\end{frame}



\begin{frame}
\frametitle{Twister-Pair: TP-Leitungen}
\end{frame}










\end{document}
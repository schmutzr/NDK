%\documentclass[a4paper]{article}
%\usepackage{beamerarticle}

\documentclass[ignorenonframetext]{beamer}
\usepackage{beamerthemesplit}
\usepackage{amssymb}

\usepackage{../fhnw-beamer}
\usepackage{fancyvrb}
\usepackage{hyperref}
\hypersetup{
    colorlinks=true,
    linkcolor=blue,
    filecolor=magenta,      
    urlcolor=cyan,
}
\urlstyle{same}

%\mode<article>{\usepackage{fullpage}}
%\mode<presentation>{\usetheme{Berlin}}

\date{\today}
\author{rolf.schmutz@fhnw.ch}
\institute{FHNW}
\title{Netzwerke und Kommunikation\\B-LS-MI 004\\Basisdienste\\LDAP/ActiveDirectory Verzeichnisdienst}


\begin{document} % ===============================================================

\section{NDK 02-050: LDAP}



\begin{frame}
\titlepage
\end{frame}



\begin{frame}
\frametitle{Ziele}
\begin{itemize}
	\item{Sie kennen die Aufgaben von LDAP}
	\item{Sie k\"onnen gezielt LDAP-Server abfragen}
\end{itemize}
\end{frame}



\begin{frame}
\frametitle{LDAP: DNS f\"ur Personen und Inventar}
LDAP steht f\"ur ``Lightweight Directory Access Protocol''\footnote{wie immer bei IETF wenn etwas ``simple'' oder ``lightweight'' heisst, ist es das nicht\ldots}
\\
\vspace{0.25cm}
LDAP ist ein flexibler Verzeichnisdienst und z.B. verwendet f\"ur
\begin{itemize}
	\item Personenverzeichnis innerhalb einer Organisation (``Telefonbuch'')
	\item Computer-Inventar
\end{itemize}
LDAP bietet im Gegensatz zu DNS auch eine Authentisierung\footnote{gesicherte Identifizierung} und wird f\"ur Autorisierung\footnote{Berechtigungen} verwendet
\end{frame}

\begin{frame}
\frametitle{LDAP: Factlets}
\begin{itemize}
	\item LDAP basiert auf dem ISO/OSI X.500 Standard
	\item LDAP wird meistens \"uber TCP\footnote{Port 389} implementiert
	\item LDAP bietet ein \href{https://ldap.com/object-classes/}{Objektmodell} wobei jeder Eintrag von mehreren Klassen ``erben'' kann -- damit werden die vorhandenen Felder\footnote{z.B. objectClass=person $\rightarrow$ sn=Familienname, givenName=Vorname, mail, \ldots} bestimmt
	\item anders als DNS wird LDAP nicht global/verteilt betrieben\footnote{obschon das
	in X.500 so vorgesehen war}, d.h. die einzelnen Verzeichnisdienste m\"ussen dem Anfrager bekannt sein
	\item es sind ``Federations'' m\"oglich -- eine Verlinkung auf andere Verzeichnisse (z.B. FHNW und Switch/AAI)
	\item die Eintr\"age sind hierarchisch\footnote{\"uber Attribute \texttt{dc} (domain-component), \texttt{ou} (organisational-unit), \texttt{l} (location), etc} geordnet, meistens nach Unternehmensstruktur
	\item es sind Beziehungen zwischen Objekten m\"oglich, das wird vorallem f\"ur Gruppenzugeh\"origkeit\footnote{memberOf} benutzt
\end{itemize}
\end{frame}



\begin{frame}
\frametitle{Implementierungen}
LDAP ist wie DNS auch ein eher ``verborgener'' Dienst, der von Endbenutzer nicht
direkt verwendet wird.

\vspace{0.25cm}
Serverseitig gibt es zwei wesentliche Implementierungen
\begin{itemize}
	\item \href{https://openldap.org}{\textbf{openLDAP}}
	\item \textbf{AD}/ActiveDirectory von Microsoft
\end{itemize}
\vspace{0.5cm}
Clientseitig wird meistens durch eine Benutzeranwendung\footnote{d.h. \"ahnlich wie eine URL/Hostnamen im Browser \"uber DNS aufgel\"ost wird} (Mail, Anmeldung, etc) und/oder
spezielle Bibliotheken kommuiziert.
\vspace{0.5cm}
Einige spezielle LDAP-Clients (GUI, command-line) sind ebenfalls vorhanden, wir
benutzen die ldap-utils von OpenLDAP
\end{frame}


\begin{frame}
\frametitle{Recordformat (1/2)}
Das verwendete (externe) Datenformat ist \href{https://en.wikipedia.org/wiki/LDAP_Data_Interchange_Format}{LDIF}
\begin{itemize}
  \item einzelne ``Records''\footnote{anders als bei DNS sind das komplette Eintr\"age/Nodes mit mehreren Attributen -- die im DNS den Resource-Records/RR entsprechen} werden durch Leerzeile oder ``--'' getrennt
  \item einzelne Attribute werden mit: \texttt{Attribut-Typ: Wert} aufgef\"uhrt
  \item ein Wert kann \"uber mehrere Zeilen gehen, die Folgezeilen m\"ussen dann einger\"uckt werden
  \item Werte k\"onnen base64 codiert sein: \texttt{sn:: S8O8dHRuZXI=}\footnote{wird mit ``::'' anstatt ``:'' als Attribut-Trenner signalisiert, vorallem f\"ur ``lange'' Werte wie Benutzerbild, Zertifikate, Namen mit Umlauten, etc} (``K\"uttner'')
  \item mindestens werden die Attribute \texttt{objectClass}\footnote{kann mehrfach vorkommen}, \texttt{dn} ``distingushed name''/Pfad und \texttt{cn} ``common name''/Name erwartet
\end{itemize}
\end{frame}

\begin{frame}[fragile]
\frametitle{Recordformat (2/2)}
\begin{tiny}
\begin{Verbatim}
dn: cn=1000002, ou=Payroll, dc=willeke, dc=com
objectClass: top
objectClass: person
objectClass: organizationalPerson
objectClass: inetOrgPerson
cn: 1000002
cn: KayC
sn: Kay
description: This is Chicky Kay's description
l: Alameda
ou: Payroll
postalAddress: Payroll$Alameda
telephoneNumber: +1 213 993-9665
title: Elite Payroll Assistant
uid: KayC
givenName: Chicky
mail: KayC@ns-mail4.com
departmentNumber: 8982
employeeType: Contract
roomNumber: 9562
secretary: cn=100135, ou=Human Resources, dc=willeke, dc=com
manager: cn=100474, ou=Administrative, dc=willeke, dc=com 
\end{Verbatim}
\end{tiny}
\end{frame}



\begin{frame}[fragile]
\frametitle{Abfragen}
LDAP bietet eine flexible textbasierte \href{https://ldap.com/ldap-filters/}{Abfragesprache} nach einem funktionalen Ansatz:
\begin{itemize}
  \item Suche nach einem Namen: \texttt{(givenName=Fritz)}
  \item Suchen k\"onnen mit ``\&'' (and), ``$\vert$'' (or) und ``!'' (not) verkn\"upft werden\footnote{dabei wird eine super-coole (LISP-like) Pr\"afixnotation verwendet: \texttt{(\& (a) (b) (c))}} -- mit zwei oder mehr Argumenten
  \item Suche nach mehreren Argumenten: \\konjunktiv: \texttt{(\& (givenName=Fritz) (sn=Knob) )} \\
  disjunktiv: \texttt{($\vert$ (objectClass=person) (objectClass=top) )} 
  \item Suche mit Platzhalter: \texttt{(sn=K*)}
  \item ``unscharfe''\footnote{wird nicht von allen Server unterst\"utzt} (approximate) Suche: \texttt{(givenName$\sim=$Franz)}
  \item \ldots und alles kann noch ``verschachtelt'' werden :)
\end{itemize} 
\end{frame}


\begin{frame}
\frametitle{Exercises}
\begin{itemize}
  \item Verwenden sie die Shell/Terminal auf \url{https://fhnw.netlabs.ch/}
  \item benutzen sie einen der \href{https://ldapwiki.com/wiki/Public\%20LDAP\%20Servers}{Public LDAP Server}\footnote{bitte nicht ``bombardieren'' sonst werden wir gesperrt}
  \item Beispiel: \texttt{ldapsearch -x -H ldap://x500.bund.de '(sn=K*)'}
  \item mit \texttt{ldapsearch -LLL -x\ldots} wird die Ausgabe ein wenig besser lesbar
\end{itemize} 
Suchen Sie:
\begin{itemize}
  \item alle Eintr\"age mit Vorname ``Fritz'' (\texttt{x500.bund.de})
  \item Eintr\"ge mit einem Nachname beginnend mit ``K'' und Vorname beginnend mit ``M'' (\texttt{x.500.bund.de})
  \item Eintr\"age mit einem \texttt{userCertificate} Attribut (\texttt{x500.bund.de})
  \item Eintr\"age mit einem \texttt{collectiveOrganizationalUnitName} Attribut (\texttt{x500.bund.de})
\end{itemize} 
\end{frame}

















\end{document}
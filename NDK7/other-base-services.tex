%\documentclass[a4paper]{article}
%\usepackage{beamerarticle}

\documentclass[ignorenonframetext]{beamer}
\usepackage{beamerthemesplit}
\usepackage{amssymb}

\usepackage{../fhnw-beamer}
\usepackage{fancyvrb}
\usepackage{verbatim}
\usepackage{hyperref}
\hypersetup{
    colorlinks=true,
    linkcolor=blue,
    filecolor=magenta,      
    urlcolor=cyan,
}
\urlstyle{same}

%\mode<article>{\usepackage{fullpage}}
%\mode<presentation>{\usetheme{Berlin}}

\date{\today}
\author{rolf.schmutz@fhnw.ch}
\institute{FHNW}
\title{Netzwerke und Kommunikation\\B-LS-MI 004\\Kurze \"Ubersicht \"uber weitere Basisdienste\ldots}


\begin{document} % ===============================================================

\section{B-LS-MI 004: more base-services}



\begin{frame}
\titlepage
\end{frame}



\begin{frame}
\frametitle{Ziele}
\begin{itemize}
	\item{Sie kennen die Aufgaben der einzelnen Dienste}
\end{itemize}
\end{frame}



\begin{frame}
\frametitle{\ldots weitere Basisdienste}
weitere {\em wichtige} Basisdienste, die praktisch universell zu finden sind:
\\
\vspace{0.25cm}
\begin{itemize}
	\item Fileserver/NAS: \href{https://en.wikipedia.org/wiki/Server_Message_Block}{CIFS/SMB} (IBM/Microsoft\footnote{SMB ursp\"unglich von IBM, CIFS eine Implementierung von SMB von Microsoft}), \href{https://en.wikipedia.org/wiki/Network_File_System}{NFS}\footnote{in typischen ``B\"uroumgebungen'' seltener zu finden, in der \href{https://en.wikipedia.org/wiki/Supercomputer}{HPC}-Welt aber noch stark vertreten. Wurde in den 2000er Jahren stark modernisiert (TCP, URLs, etc)} (Sun/Oracle)\\Common Internet Files Service wird praktisch universell unterst\"utzt, Network File Service vorallem von UNIX-\"ahnlichen Systemen
	\item Zeitsynchronisation: \href{http://support.ntp.org/bin/view/Main/WebHome}{NTP}\footnote{ein nicht zu untersch\"atzender Dienst: z.B. Authetisierung via Kerberos/AD funktioniert mit engen Zeitfenster} Network Time Protocol\\erreicht Zeitsynchronisation im Millisekundenbereich (Internet)
	\item Device-Control: \href{https://en.wikipedia.org/wiki/}{SNMP}\footnote{note: IETF $\rightarrow$ ain't ``simple''} Simple Network Management Protocol\\Steuern und \"Uberwachen von Router, Switches, Drucker, etc
	\item Drucker: \href{https://en.wikipedia.org/wiki/Internet_Printing_Protocol}{IPP}/\href{https://en.wikipedia.org/wiki/CUPS}{CUPS} 631/tcp, \href{https://en.wikipedia.org/wiki/Line_Printer_Daemon_protocol}{lpr} 515/tcp, \href{https://en.wikipedia.org/wiki/JetDirect}{jetdirect} 9100/tcp\\
	die meisten modernen Netzwerk-Drucker unterst\"utzen alle Transportmechanismen
\end{itemize}
\end{frame}


\begin{frame}
\frametitle{Allgemeines}
\begin{itemize}
	\item CIFS, NFS, NTP und SNMP werden \"uber UDP\footnote{effizienter in einer Umgebung mit kleinem ``packet-loss'' -- typischerweise Campus/Firmen-Netzwerke} und TCP angeboten
	\item IPP, lpr, jedirect (die Druckerprotokolle) meistens \"uber TCP
	\item alle Protokolle sind ``bin\"ar'', also mit einem nicht-menschenlesbaren Format\footnote{allgemein kompakter und einfacher zu implementieren in embedded-devices/kleinen Comp\"uterli}
	\item ausser IPP sind alle ``uralt'' -- sp\"ate 80iger, fr\"uhe 90iger Jahre
	\item die Protokolle werden von jedem aktuellen Betriebssystem unterst\"utzt
\end{itemize}
\end{frame}



\begin{frame}
\frametitle{Fileservices}
\ldots heute popul\"arer als NAS -- Network Attached Storage bezeichnet
\begin{itemize}
	\item erlaubt das ``mounten'' eines entfernten Filesystems in die lokale FS-Hierarchie\footnote{die alte Windows-Technik des Laufwerk-mappings wird weiterhin unterst\"utz}
	\item bieten File-Access auf Block-Level\footnote{anders als FTP/WebDAV, das File{\em transfer} implementiert}\\
	also wie ein lokales Filesystem auf der Festplatte mit der bekannten
	Zugriffsemantik \texttt{open}, \texttt{read}, \texttt{write}, \texttt{seek}, \texttt{close}
	\item File-Locking Mechanismen um gleichzeitigen Zugriff zu erm\"oglichen
\end{itemize}
\end{frame}



\begin{frame}[fragile]
\frametitle{Printing}
\begin{itemize}
	\item von eher primitiv (jetdirect, lpr) bis ausgekl\"ugelt (IPP)
	\item zus\"tzlich zum reinen ``drucken'' werden auch Statusabfragen/Steuerbefehle implementiert\footnote{printer-queue Verwaltung, Modusumschaltung, etc}
	\item \ldots eher ``langweiliger'' Dienst aber unverzichtbar :)
	\item Beispiel der Dienste\footnote{http/s wird zur Verwaltung benutzt, FTP tats\"achlich zum Drucken} eines ``modernen'' Netzwerkdruckers
	\end{itemize}
\begin{center}
	\begin{tiny}
	\begin{Verbatim}
PORT     STATE SERVICE
21/tcp   open  ftp
80/tcp   open  http
443/tcp  open  https
515/tcp  open  printer
631/tcp  open  ipp
9100/tcp open  jetdirect
161/udp  open  snmp
\end{Verbatim}
\end{tiny}
\end{center}
\end{frame}



\begin{frame}
\frametitle{Zeitsynchronisation}
\begin{itemize}
    \item hierarchische Zeitserver\footnote{Stratum 0 sind direkte C\"asium-Zeitquellen, Stratum 1 GPS-normale, etc}, Einweg- oder Mehrweg-Synchronisierung\footnote{client/server oder ``peer-to-peer``}
    \item Links: \href{http://support.ntp.org/bin/view/Main/WebHome}{offizielle NTP Seite}\footnote{das HTML-Design ist in den 90iger Jahren stehengeblieben -- ein Tribut an die Effizienz}, \href{https://www.metas.ch/metas/en/home/fabe/zeit-und-frequenz/time-dissemination.html}{offizielle Zeitquelle des Bundes} (CH), \href{https://www.ntppool.org/}{NTP-Pool}, falls keine voreingestellte Zeitquelle eingerichtet ist

	
	\item Beispiel: (\texttt{ntpq -p})
\end{itemize}
\begin{center}
\begin{tiny}
\verbatiminput{ntpq-sample.txt}
\end{tiny}
\end{center}

\end{frame}




\begin{frame}
\frametitle{\href{https://en.wikipedia.org/wiki/Simple_Network_Management_Protocol}{SNMP}}
\begin{itemize}
    \item erweiterbares Management-Protokoll (Schemas/MIB)
	\item implementiert\footnote{``Agent''} auf den meisten Betriebssystemen, Netzwerk-Drucker, Netzwerk-Ger\"ate wie Router, Switches (``managed'')
	\item \href{http://www.net-snmp.org}{net-snmp}, die oft verwendete ``Basis''-Implementation
	\item Beispiel eines Schemas: (\texttt{snmpwalk -c public -v1 192.168.1.87})
\end{itemize}
\begin{center}
\begin{tiny}
\verbatiminput{brother-snmp-sample.txt}
\end{tiny}
\end{center}
\end{frame}

















\end{document}